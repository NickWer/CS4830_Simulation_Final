\documentclass{article}

\usepackage{amsmath}
\usepackage[margin=1in]{geometry}
\usepackage{minted}
\usepackage{multicol}
\usepackage{verbatim}
\usepackage{authblk}
\usepackage{graphicx}
\usepackage[T1]{fontenc}

\renewcommand\Affilfont{\itshape\small}
\newcommand{\srcFile}[2]{
    \subsection{#1}
    #2
    \inputminted[linenos,breaklines]{csharp}{../#1}
    \newpage
}

\newcommand{\img}[2]{
	\begin{figure}[h!]
		\includegraphics{img/#1}
		\centering
		\caption{#2}
	\end{figure}
}

\title{CS4830 System Simulation Final Project}
\author{Nick Werle}

\begin{document}
\maketitle
\newpage
\tableofcontents
\newpage
\section{Introduction}
The year is 2075, and robots mining on mars is now a thing.
MineCo, the biggest player in the Mars based robot mining game, has decided that to answer business questions, they should be using simulations rather than their current methodology of just guessing.
This document will enumerate the simulation methodology, some of the models used in the simulation, and then 3 questions and the answers that the simulation produced.

\section{Discussion}
\subsection{Simulation Background}
This simulation has a few basic components.
First, there is one mining base.
The mining base has chargers that can be used by any robot, but no more than any one robot.
Next, there are mining sites, which can only be mined by one robot at a time.
Finally, there are robots, that do the mining.
Robots have a speed that is used to determine how long it takes to get to the mining site or back to the base.
Robots have an unloading time, and they can have different digging implements that mine at different speeds.
Additionally, robot batteries have variable qualities, which affect how well they charge, and the power consumed is a function of amount of ore being carried.

\subsection{Simulation Methodology}
The simulation follows a simple structure.
\end{document}
